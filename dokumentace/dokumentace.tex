\documentclass[a4paper,12pt]{article}
\usepackage{tabularx}
\usepackage{times}
\usepackage[hidelinks]{hyperref}
\hypersetup{unicode}
\usepackage[czech]{babel}
\usepackage[utf8]{inputenc}
\usepackage[IL2]{fontenc}
\usepackage[left=2cm,top=3cm,text={17cm, 24cm}]{geometry}
\begin{document}

\setlength{\tabcolsep}{0.6em}
\renewcommand{\arraystretch}{1.5}

\begin{titlepage}
    \begin{center}
        \textsc{\Huge Vysoké učení technické v Brně \medskip}\\
        \textsc{\huge Fakulta informačních technologií} \\
        \vspace{\stretch{0.382}}
        {\huge Formální jazyky a překladače \medskip}\\
        {\Huge Překladač imperativního jazyka IFJ19 \medskip} \\
        {\Large Tým 052, varianta II}
        \vspace{\stretch{0.618}}
        \begin{table}[b]
            \large
            \begin{tabularx}{\textwidth}{l X r l l}
                \textbf{Vedoucí:} &  & \textbf{Jaroslav Hort} & \textbf{(xhortj04)} & \textbf{25\%} \\ 
                &  & Filip Dráber & (xdrabe09) & 25\% \\ 
                &  & Iveta Strnadová & (xstrna14) & 25\% \\ 
                &  & Norbert Pócs & (xpocsn00) & 25\% \\ 
                \today &  &  & \\ 
            \end{tabularx}
        \end{table}
    \end{center}
\end{titlepage}

\setlength{\tabcolsep}{0.8em}
\renewcommand{\arraystretch}{1.7}

\section{Generátor výsledného kódu}

Po úspěsňé analýze je inicializován generátor výsledného kódu. Součástí inicializace 
je vyvtoření tabulek s rozptýlenými položkami pro ukládání názvu vytvořených
proměnných a labelů pro funkce. Vlastní spuštění generátoru spočívá ve volání
funkce \texttt{generate\_code}.

\subsection{Tří adresný kód}

Tří adresný kód je generován na úrovni parseru a precedenční analýzy. Jeho struktura
obsahuje v typ kódu, \texttt{ac\_type}, a tři operandy typu \texttt{Token}, \texttt{op1}, \texttt{op2}, \texttt{res}.
Výsledná struktura kódu je uložena do seznamu kódu, který je využit v generátoru. Hlavními funkcemi
seznamu jsou funkce \texttt{setACAct}, která nastaví aktivitu seznamu na první položku,
\texttt{actAC}, pro nastavení aktivity na následující položku a \texttt{isACActive} pro
zjištení, zda je v seznamu aktivní prvek.

\subsection{Průběh generování}

Ze seznamu tři adresných kódu jsou postupně brány položky a podle typu kódu je
rozhodnuto jaká instrukce bude zapsána do výsledného kódu. Výsledný kód je v 
průběhu generování rozdělen na dva bloky, kód hlavního programu (main) a kód funkcí a
ukládán do dynamicého řetězce pro tyto bloky.

Tento způsob implementace byl zvolen z důvodu, že volání funkcí je realizováno pomocí
skoku na návěští, pokud by byly funkce zapsány uvnitř hlavního kódu, bylo by nutné
je přeskakovat, dokud nebudou volány, což by bylo horší na implementaci a výsledný kód
by byl méně přehledný.

Po přečtení všech kódu a vygenerován instrukcí je výsledek vypsán na standardní výstup,
nejprve kód funkci a poté kód hlavního bloku programu.

\subsection{Vestavěné funkce}

Vestavěné funkce jsou definovány v hlavičkovém souboru generátoru. V případě, že program poprvé volá
jednu z těchto funkcí, jsou zapsnány do bloku funkcí a funkce je volána jako obyčejné uživatelské 
funkce. Výsledný kód tedy obsahuje pouze ty funkce, které jsou využívány, tímto rešením udržíme kód 
relativně čistý.

\subsection{Generování identifikátorů}

Při generování instrukcí pracujících s identifikátory je nejprve prohledána tabulka,
zda nebyl identifikátor generován již dříve. Identifikátory z hlavního bloku programu,
takzvané globání identifikátory, mají přednost před identifikátory uvnitř funkcí, nebo-li lokálními
identifikátory. Pokud nebyl nalezen, je zapsána insturkce do kódu a do příslušné tabulky je 
vloženo jméno identifikátoru

\subsection{Problém generování cyklu}

U generování cyklu \texttt{while} byl zjištěn problém možné definice identifikátoru. Tuto možnost
jsme ošetřili tak, že parser vygeneruje tří adresný kód \texttt{WHILE\_START} pro označení začátku cyklu
a \texttt{WHILE\_END} pro konec cyklu. Když generátor narazí na začátek cyklu, projde všechny kódy
uvnitř cyklu, provede kontrolu definice identifikátorů a případně identifikátory nadefinuje předem,
než cyklus začne.

\section{Pomocné knihovny}

\subsection{Knihovna chyb \texttt{errors}}

Knihovna obsahuje výčet kódu všech možných chyb, které mohou nastat při překladu programu, globání
proměnnou \texttt{global\_error\_code}, kterou jednotlivé moduly nastavují na jeden z typu chyby 
v případě, že někde nastala. Na této proměnné zavisí spuštění generátoru. Dále máme dvě funkce
pro výpis chyby na standardní chybový výstup \texttt{stderr}, první pro výpis interní chyby a druhá
pro kompilační chyby, která tiskne navíc název souboru a číslo řádku zdrojového programu 
na které se chyba vyskytuje.

\subsection{Dynamický řetězec}

Pro jednodušší práci s řetězci v programovacím jazyce C jsme vytvořili knihovnu \texttt{dynamic\_string}.
Knihovna nám umožnuje vytvořit nový řetězec, rozšířit ho připsáním nového řetězce na konec stávajícího,
nebo ho skrátit na požadovanou délku.

\section{Práce v týmu}

Pro jednoduchou práci v týmu jsme zvolili verzovací systém Git a náš repozitář umístili na server GitHub.
Každý z nás měl vlastní větev, ve které psal svůj kód a dle potřeby ji sloučil s hlavní větví.
Komunikace probíhala osobně v podobě schůzek v prostrách fakulty, nebo online před komunikační 
kanál Discord.

\subsection{Rozdělení}

\begin{table}[h]
    \large
    \begin{tabularx}{\textwidth}{l|X|r}
        \textbf{Student} & \textbf{Práce} & \textbf{\%} \\ \hline
        Jaroslav Hort & Generátor kódu, rozhraní, knihovny pro dynamický řetězec, chyby, tabulky rámců, dokumentace, prezentace & 25\% \\ 
        Filip Dráber & Precedenční analýza, dokumentace & 25\%  \\ 
        Iveta Strnadová & Parser, LL tabulka, tabulka symbolů, tří adresný kód, knihovna pro pro praci s \texttt{indent\_stack}, dokumentace & 25\% \\ 
        Norbert Pócs & Lexikální analyzátor, knihovna pro token, dokumentace & 25\% \\ 
    \end{tabularx}
\end{table}

\end{document}